\documentclass[a4paper,12pt]{article}
\usepackage[portuguese]{babel}
\usepackage[utf8]{inputenc}
\usepackage[normalem]{ulem}
\usepackage{color}
\title{A vida do Suricata}
\author{Alvinho Rodrigues Ing das Neves}
\date{\today}
\begin {document}
\maketitle

\section{Sobre o Suricata}
O \textbf{suricata},  também chamado de \textbf{suricato} ou suricate(\textit{Suricata} \textit{suricatta}) é um pequeno mamífero da \textit{Herpestidae}, nativo do deserto Kalahari. Estes animas tem cerca de meio metro de comprimento ( incluindo a cauda), em meia 730 gramas de de peso e, pelagem acastanhada.Tem garras afiadas na pata, que lhes permitem escavar a superfície  de chão e dentes afiados para penetrar nas carapaças quintinosas das suas presas.\textcolor{red}{Outra característica distinta é a sua capacidade de se elevarem nas patas traseiras,utilizando a cauda como terceiro apoio}.
\section{Características Gerais}

\subsection{Alimentação}

Alimenta-se principalmente de insetos (cerca de 82\% ): larvas de escaravelhos e de borboletas: também ingerem milípedes, aranha, escorpiões, pequenos vertebrados( répteis, anfíbios e aves),ovos e matéria vegetal.\uline{Se são relativamente imunes ao veneno das najas} e dos escorpiões, sendo estes, inclusive, um dos alimentos que mais apreciam.


 

\end{document}

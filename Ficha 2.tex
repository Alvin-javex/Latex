\documentclass[a4paper,11pt]{report}
\usepackage[utf8]{inputenc}
\usepackage[normalem]{ulem}
\usepackage{xcolor}
\usepackage[portuguese]{babel}
\usepackage{colortbl}


\title{A vida do Suricata}
\author{Alvinho Rodrigue Ing das Neves}
\date{\today}

\begin{document}
\maketitle
\pagenumbering{arabic}
\chapter{Introdução}
\section{Sobre o Suricata}
O \textbf{suricata}, também chamado de  \textbf{suricato} ou  \textbf{suricate} \textit{(\textbf{Suricata} su-
ricatta)} é um pequeno mamífero da família \textit{Herpestidae}, nativo do
deserto do Kalahari. Estes animais têm cerca de meio metro de com-
primento (incluindo a cauda), em média 730 gramas de peso, e pela-
gem acastanhada. Têrm garras afiadas nas patas, que lhes permitem
escavar a superfície do chão e dentes afiados para penetrar nas ca-
rapaças quitinosas das suas presas. \textcolor{red}{Outra característica distinta é a sua capacidade de se elevarem nas patas traseiras, utilizando a cauda como terceiro apoio.}

\section{Características gerais}
\subsection{Alimentação}
Alimenta-se principalmente de insetos (cerca de 82\%):
\begin{itemize}
\item larvas de escaravelhos de insetos e de borboletas;
\item milípedes;
\item aranhas.
\end{itemize}

\begin{itemize}
\item escorpiões;
\item pequenos; vertebrados( répteis, anfíbios e aves);
\item ovo;
\item matérial vegetal.
\end{itemize}
\uline {São relativamente imunes ao veneno} das najas  e dos escorpiões, sendo  estes, inclusive, um dos alimentos que  mais apreciam.
\chapter{Desenvolvimento}
\section{Onde avistar  suricatas no habitat selvagem?}

Existem vários nacionais em África onde é possível avistar e até interagir com suiricatas no seu habitat selvagem. No entanto, existe uma regra de ouro: os suricatas não gostam de chuva, por isso prefira dias solarengos.
\\  Em baixo apresenta.se uma lista de parques ordenada por números de suricatas por Km²:
\begin{enumerate}
\item Parque "Kgalagadi",África do Sul /Botswana
\item Parque nacional "Karoo", África do Sul
\item Reserva do vale mágico do Suricata, África do Sul
\item Parque nacional Iona, Angola
\end{enumerate}
ou

\begin{description}
\item[parque1] Parque nacinal " Karoo", África do Sul / Botwana
\item[parque2] Parque nacional " Karoo", África do sul
\item[parque3] Reserva do vale, mágico do Suricata,Àfrica do Sul
\item[parque4] Parque nacioanal, Angola
\end{description}
\section{Subespécies}
Existem altualmente três subespécies de Suricata:
\begin{itemize}
\item Suricata suricatta siricata;
\item Suricata suricatta iona;
\item Suricata suricatta majoriae.

\end{itemize}
\newpage
Os individuos de cada subspécie apresentam características distintas   como se pode ver na tabela 2.1.

\begin{table}[h!] 
\begin{tabular}{|c|l|l|l|}
\hline
\rowcolor{gray!40}
\multicolumn{1}{c|}{} &
\multicolumn{1}{c|}{\textbf{siricata}} &
\multicolumn{1}{c|}{\textbf{iona}} &
\multicolumn{1}{c}{\textbf{majoriae}}\\ \hline
\textbf{côr do pelo} & beje amarelo & castanho amarelado & castanho escuro\\ \hline
\textbf{tamanho} & 29 cm & 25 cm &  34 cm \\ \hline
\textbf{peso} & 731g & 698g & 799 \\ \hline

\end{tabular}
\caption{Caracteristicas diferenciadoras entre subespécies de Suricata.}

\end{table}


\end{document}
